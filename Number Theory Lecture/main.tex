\documentclass[oneside]{book}

\usepackage{amsmath, amsthm, amssymb, amsfonts}
\usepackage{thmtools}
\usepackage{graphicx}
\usepackage{setspace}
\usepackage{geometry}
\usepackage{float}
\usepackage{hyperref}
\usepackage[utf8]{inputenc}
\usepackage[english]{babel}
\usepackage{framed}
\usepackage[dvipsnames]{xcolor}
\usepackage{environ}
\usepackage{tcolorbox}
\usepackage{algorithm}
\usepackage{algpseudocode}
\usepackage{csquotes}
\usepackage[backend=biber, style=ieee]{biblatex}
\tcbuselibrary{theorems, skins, breakable}
\addbibresource{references.bib}


\setstretch{1.2}
\geometry{
    textheight=9in,
    textwidth=5.5in,
    top=1in,
    headheight=12pt,
    headsep=25pt,
    footskip=30pt
}

% Variables
\def\notetitle{Intro to Number Theory}
\def\noteauthor{
    \textbf{Lecture} and\\ 
    {\LaTeX} by Eric Xingyu Yang\\
    Monash University\\
	\textcite{rosenDiscreteMathematicsIts2018,rosenElementaryNumberTheory2010} are the main references
	}
\def\notedate{Semester one 2025}

% The theorem system and user-defined commands
% Theorem System
% The following boxes are provided:
%   Definition:     \defn 
%   Theorem:        \thm 
%   Lemma:          \lem
%   Corollary:      \cor
%   Proposition:    \prop   
%   Claim:          \clm
%   Fact:           \fact
%   Notation:       \notn
%   Problem:        \prob
%   Solution:       \soln
%   Proof:          \pf
%   Example:        \ex
%   Remark:         \rmk (sentence), \rmkb (block)
% Suffix
%   r:              Allow Theorem/Definition to be referenced, e.g. thmr
%   p:              Add a short proof block for Lemma, Corollary, Proposition or Claim, e.g. lemp
%                   For theorems, use \pf for proof blocks

% Definition
\newtcbtheorem[number within=section]{mydefinition}{Definition}
{
    enhanced,
    frame hidden,
    titlerule=0mm,
    toptitle=1mm,
    bottomtitle=1mm,
    fonttitle=\bfseries\large,
    coltitle=black,
    colbacktitle=green!20!white,
    colback=green!10!white,
}{defn}

\NewDocumentCommand{\defn}{m+m}{
    \begin{mydefinition}{#1}{}
        #2
    \end{mydefinition}
}

\NewDocumentCommand{\defnr}{mm+m}{
    \begin{mydefinition}{#1}{#2}
        #3
    \end{mydefinition}
}

% Theorem
\newtcbtheorem[use counter from=mydefinition]{mytheorem}{Theorem}
{
    enhanced,
    frame hidden,
    titlerule=0mm,
    toptitle=1mm,
    bottomtitle=1mm,
    fonttitle=\bfseries\large,
    coltitle=black,
    colbacktitle=cyan!20!white,
    colback=cyan!10!white,
}{thm}

\NewDocumentCommand{\thm}{m+m}{
    \begin{mytheorem}{#1}{}
        #2
    \end{mytheorem}
}

\NewDocumentCommand{\thmr}{mm+m}{
    \begin{mytheorem}{#1}{#2}
        #3
    \end{mytheorem}
}

% Lemma
\newtcbtheorem[use counter from=mydefinition]{mylemma}{Lemma}
{
    enhanced,
    frame hidden,
    titlerule=0mm,
    toptitle=1mm,
    bottomtitle=1mm,
    fonttitle=\bfseries\large,
    coltitle=black,
    colbacktitle=violet!20!white,
    colback=violet!10!white,
}{lem}

\NewDocumentCommand{\lem}{m+m}{
    \begin{mylemma}{#1}{}
        #2
    \end{mylemma}
}

\newenvironment{lempf}{
	{\noindent{\it \textbf{Proof for Lemma}}}
	\tcolorbox[blanker,breakable,left=5mm,parbox=false,
    before upper={\parindent15pt},
    after skip=10pt,
	borderline west={1mm}{0pt}{violet!20!white}]
}{
    \textcolor{violet!20!white}{\hbox{}\nobreak\hfill$\blacksquare$} 
    \endtcolorbox
}

\NewDocumentCommand{\lemp}{m+m+m}{
    \begin{mylemma}{#1}{}
        #2
    \end{mylemma}

    \begin{lempf}
        #3
    \end{lempf}
}

% Corollary
\newtcbtheorem[use counter from=mydefinition]{mycorollary}{Corollary}
{
    enhanced,
    frame hidden,
    titlerule=0mm,
    toptitle=1mm,
    bottomtitle=1mm,
    fonttitle=\bfseries\large,
    coltitle=black,
    colbacktitle=orange!20!white,
    colback=orange!10!white,
}{cor}

\NewDocumentCommand{\cor}{+m}{
    \begin{mycorollary}{}{}
        #1
    \end{mycorollary}
}

\newenvironment{corpf}{
	{\noindent{\it \textbf{Proof for Corollary.}}}
	\tcolorbox[blanker,breakable,left=5mm,parbox=false,
    before upper={\parindent15pt},
    after skip=10pt,
	borderline west={1mm}{0pt}{orange!20!white}]
}{
    \textcolor{orange!20!white}{\hbox{}\nobreak\hfill$\blacksquare$} 
    \endtcolorbox
}

\NewDocumentCommand{\corp}{m+m+m}{
    \begin{mycorollary}{}{}
        #1
    \end{mycorollary}

    \begin{corpf}
        #2
    \end{corpf}
}

% Proposition
\newtcbtheorem[use counter from=mydefinition]{myproposition}{Proposition}
{
    enhanced,
    frame hidden,
    titlerule=0mm,
    toptitle=1mm,
    bottomtitle=1mm,
    fonttitle=\bfseries\large,
    coltitle=black,
    colbacktitle=yellow!30!white,
    colback=yellow!20!white,
}{prop}

\NewDocumentCommand{\prop}{+m}{
    \begin{myproposition}{}{}
        #1
    \end{myproposition}
}

\newenvironment{proppf}{
	{\noindent{\it \textbf{Proof for Proposition.}}}
	\tcolorbox[blanker,breakable,left=5mm,parbox=false,
    before upper={\parindent15pt},
    after skip=10pt,
	borderline west={1mm}{0pt}{yellow!30!white}]
}{
    \textcolor{yellow!30!white}{\hbox{}\nobreak\hfill$\blacksquare$} 
    \endtcolorbox
}

\NewDocumentCommand{\propp}{+m+m}{
    \begin{myproposition}{}{}
        #1
    \end{myproposition}

    \begin{proppf}
        #2
    \end{proppf}
}

% Claim
\newtcbtheorem[use counter from=mydefinition]{myclaim}{Claim}
{
    enhanced,
    frame hidden,
    titlerule=0mm,
    toptitle=1mm,
    bottomtitle=1mm,
    fonttitle=\bfseries\large,
    coltitle=black,
    colbacktitle=pink!30!white,
    colback=pink!20!white,
}{clm}


\NewDocumentCommand{\clm}{m+m}{
    \begin{myclaim*}{#1}{}
        #2
    \end{myclaim*}
}

\newenvironment{clmpf}{
	{\noindent{\it \textbf{Proof for Claim.}}}
	\tcolorbox[blanker,breakable,left=5mm,parbox=false,
    before upper={\parindent15pt},
    after skip=10pt,
	borderline west={1mm}{0pt}{pink!30!white}]
}{
    \textcolor{pink!30!white}{\hbox{}\nobreak\hfill$\blacksquare$} 
    \endtcolorbox
}

\NewDocumentCommand{\clmp}{m+m+m}{
    \begin{myclaim*}{#1}{}
        #2
    \end{myclaim*}

    \begin{clmpf}
        #3
    \end{clmpf}
}

% Fact
\newtcbtheorem[use counter from=mydefinition]{myfact}{Fact}
{
    enhanced,
    frame hidden,
    titlerule=0mm,
    toptitle=1mm,
    bottomtitle=1mm,
    fonttitle=\bfseries\large,
    coltitle=black,
    colbacktitle=purple!20!white,
    colback=purple!10!white,
}{fact}

\NewDocumentCommand{\fact}{+m}{
    \begin{myfact}{}{}
        #1
    \end{myfact}
}


% Proof
\NewDocumentCommand{\pf}{+m}{
    \begin{proof}
        [\noindent\textbf{Proof.}]
        #1
    \end{proof}
}

% Example
\newenvironment{example}{%
    \par
    \vspace{5pt}
	\begin{minipage}{\textwidth}
		\noindent\textbf{Example.}
		\tcolorbox[blanker,breakable,left=5mm,parbox=false,
	    before upper={\parindent15pt},
	    after skip=10pt,
		borderline west={1mm}{0pt}{cyan!10!white}]
}{%
		\endtcolorbox
	\end{minipage}
    \vspace{5pt}
}

\NewDocumentCommand{\ex}{+m}{
    \begin{example}
        #1
    \end{example}
}

% Notation
\newenvironment{notation}{%
    \par
    \vspace{5pt}
    \begin{minipage}{\textwidth}
        \noindent\textbf{Notation.}
        \tcolorbox[blanker,breakable,left=5mm,parbox=false,
        before upper={\parindent15pt},
        after skip=10pt,
        borderline west={1mm}{0pt}{teal!30!white}]
}{%
        \endtcolorbox
    \end{minipage}
    \vspace{5pt}
}

\NewDocumentCommand{\notn}{+m}{
    \begin{notation}
        #1
    \end{notation}
}

% Problem
\newenvironment{problem}{%
    \par
    \vspace{5pt}
    \begin{minipage}{\textwidth}
        \noindent\textbf{Problem.}
        \tcolorbox[blanker,breakable,left=5mm,parbox=false,
        before upper={\parindent15pt},
        after skip=10pt,
        borderline west={1mm}{0pt}{red!20!white}]
}{%
        \endtcolorbox
    \end{minipage}
    \vspace{5pt}
}

\NewDocumentCommand{\prob}{+m}{
    \begin{problem}
        #1
    \end{problem}
}

% Problems (for multiple problems)
\newenvironment{problems}{%
    \par
    \vspace{5pt}
    \begin{minipage}{\textwidth}
        \noindent\textbf{Problems.}
        \tcolorbox[blanker,breakable,left=5mm,parbox=false,
        before upper={\parindent15pt},
        after skip=10pt,
        borderline west={1mm}{0pt}{red!30!white}]
}{%
        \endtcolorbox
    \end{minipage}
    \vspace{5pt}
}

% Solution
\newenvironment{solution}{%
    \par
    \vspace{5pt}
    \begin{minipage}{\textwidth}
        \noindent\textbf{Solution.}
        \tcolorbox[blanker,breakable,left=5mm,parbox=false,
        before upper={\parindent15pt},
        after skip=10pt,
        borderline west={1mm}{0pt}{lime!20!white}]
}{%
        \endtcolorbox
    \end{minipage}
    \vspace{5pt}
}

\NewDocumentCommand{\soln}{+m}{
    \begin{solution}
        #1
    \end{solution}
}

% Remark
\NewDocumentCommand{\rmk}{+m}{
    {\it \color{blue!50!white}#1}
}

\newenvironment{remark}{
    \par
    \vspace{5pt}
    \begin{minipage}{\textwidth}
        {\par\noindent{\textbf{Remark.}}}
        \tcolorbox[blanker,breakable,left=5mm,
        before skip=10pt,after skip=10pt,
        borderline west={1mm}{0pt}{cyan!10!white}]
}{
        \endtcolorbox
    \end{minipage}
    \vspace{5pt}
}

\NewDocumentCommand{\rmkb}{+m}{
    \begin{remark}
        #1
    \end{remark}
}


\newcommand{\lcm}{\operatorname{lcm}}
\newcommand{\Div}{\operatorname{\text{\textbf{div}}}}
\newcommand{\Mod}{\operatorname{\text{\textbf{mod}}}}



% ------------------------------------------------------------------------------

\begin{document}
\title{\textbf{
		\LARGE{\notetitle} \vspace*{10\baselineskip}}
}
\author{\noteauthor}
\date{\notedate}

\maketitle
\newpage

\tableofcontents
\newpage

% ------------------------------------------------------------------------------

\chapter{Modular Arithmetic and Algebraic Structures}
\section{Division}
One of the most fundamental concepts in number theory is the division of integers.
Whether a number is divisible by another number? How can we determine the
divisibility of two numbers? How can we express one number in terms of another?
These questions lead us to the foundation of number theory, which is one of the most
abstruse and beautiful branches of mathematics.

Number theory plays a crucial role in many fields:
\begin{itemize}
	\item \textbf{Cryptography}: The security of many encryption algorithms relies on the properties of prime numbers and modular arithmetic.
	\item \textbf{Algebraic Geometry}: Number theory is used to study the properties of algebraic varieties over finite fields. This is
	      crucial in coding theory and cryptography as well. This is also known as the hardest math branch encompassing almost all the math fields.
	      An example is
	      \href{https://cryptobook.nakov.com/asymmetric-key-ciphers/elliptic-curve-cryptography-ecc}{Elliptic Curve Cryptography(ECC)}.
	\item \textbf{Computer Science}: In computer science, number theory is used in algorithms, data structures, and complexity theory.
	      A typical example is the design of \href{https://en.wikipedia.org/wiki/Hash_function}{hash functions}.
	\item \textbf{Optimization and Operations Research}: Number theory is used in optimization problems, such as integer programming and combinatorial optimization.
	      In some cases, we can use number theory notions to simplify the problem, such as \textbf{modular congruence} and \textbf{Diophantine equations}.
\end{itemize}

It might be trivial, since we have been using division since we were kids,
but it is necessary to give division a formal definition.
\defn{True Division}{
	Formally, the division of two real numbers \( a, b \in \mathbb{R} \) is defined as:
	\[
		a \div b = c
	\]
	where \( c \) is the unique number such that:
	\[
		a = b \times c.
	\]
	In other words, $ a \div b = c \iff a = b \times c $.
	The division is defined for all \( a \in \mathbb{R} \) and \( b \in \mathbb{R} \setminus \{0\} \).
}

The division is not defined for \( b = 0 \) because it leads to an undefined result.
This is actually a very polemical topic in the history of mathematics, and gave birth to
the \textbf{division by zero} paradox, contributing to the formalization of infinity, which is
a milestone in the development of calculus and analysis.

While this is only an intro, in number theory, we are more interested in the division of integers. Why?
\fact{
	\begin{itemize}
		\item \textbf{Integers are the building blocks of all numbers}: All numbers can be expressed as a combination of integers.
		\item \textbf{Integers are discrete}: Unlike real numbers, integers are discrete and can be counted. This makes them easier to work with in many cases.
		\item \textbf{Unique factorization}: Every integer can be \textbf{uniquely} factored into other numbers, which is a fundamental property of integers.
	\end{itemize}
}


\defn{Division with Remainder}{
	The division of a number \( a \) by a non-zero number \( b \) is denoted by \( a \div b \) or \( \frac{a}{b} \), and it gives the quotient \( q \) and possibly a remainder \( r \). The operation can be written as:
	\[
		a = b \times q + r
	\]
	where \( 0 \leq r < b \) and \(q \in \mathbb{Z}\).
}

\defnr{Divisibility}{divisibility}{
	If $a$ and $b$ are integers with $a \neq 0$, we say that $a$ divides $b$ if there is an integer $c$ such that
	$b = a\times c$ (or equivalently, if $b$ an is an integer). When $a$ divides $b$ we say that $a$ is a factor or divisor of
	$b$, and that $b$ is a multiple of $a$. The notation a $\mid$ b denotes that $a$ divides $b$. We write $a \nmid b$ when $a$ does not divide $b$
}

Divisibility lays the foundation of number theory, and later we will introduce how to use it to define modular arithmetic.

\ex{
	For example, we have $4\mid 20$, because $20\div4$ gives an integer, however, $4\nmid 21$, as the answer
	cannot be represented as integer. Besides, we have:
	\begin{equation}
		a\mid b \iff \exists c (ac=b).
	\end{equation}
}

\prob{
	Prove that if $\displaystyle a$ and $\displaystyle b$ are integers and $\displaystyle a$ divides $\displaystyle b$, then $\displaystyle a$ is odd or $\displaystyle b$ is even.
}


We may find more properties of divisibility.
\prop{\begin{itemize}
		\item If \( a \mid b \) and \( b \mid c \), then \( a \mid c \).
		      \begin{itemize}
			      \item If $a \mid b$ and $a \mid c$, then $a \mid (b+c)$.
			      \item If $a \mid b$ then $a \mid bc $ for any integer $c$.
		      \end{itemize}
		\item For any integer \( a \), \( a \mid a \).
		\item If \( a \mid b \), then \( b \nmid a \) unless \( a = b \).
		\item If \( a = 0 \), then \( 0 \mid b \) for all integers \( b \).
	\end{itemize}}


\prob{
	Prove that if $a$, $b$, and $c$ are integers, where $a\neq 0$, such that $a\mid b$ and $a\mid c$, then $a\mid mb + nc$ whenever $m$ and $n$ are integers.
}


\section{Modular Arithmetic}

\subsection{Div and Mod Operator}

\notn{
	\textbf{div} and \textbf{mod} are two operators that are used to perform division and find the remainder of a division operation, respectively.
	\begin{itemize}
		\item \textbf{div}: The div operator is used to perform integer division. It returns the quotient of the division operation, discarding any remainder.
		      For example, \( 7 \div 3 = 2 \) because \( 3 \times 2 = 6 \) and the remainder is discarded.
		\item \textbf{mod}: The mod operator is used to find the remainder of a division operation. It returns the remainder after dividing one number by another.
		      For example, \( 7 \mod 3 = 1 \) because \( 7 = 3 \times 2 + 1 \).

		      We also use \% to denote the mod operator, which is more common in programming languages. But we will use \textbf{mod} in mathematical context.
	\end{itemize}
}
\rmkb{
	Note that, the \textbf{mod} operator should be bold in a printed document, since we need to differentiate it with $\mod$, which is used for modular congruence.
}

You may find that the \textbf{div} and \textbf{mod} operators are actually binary and can be encapsulated by binary equations.

\defn{Div and Mod}{
	Let \(\mathbb{Z}\) denote the set of all integers. We define two functions as follows:

	\begin{enumerate}
		\item The integer division function, denoted \(\operatorname{div} : \mathbb{Z} \times (\mathbb{Z} \setminus \{0\}) \to \mathbb{Z}\), is defined by
		      \[
			      \operatorname{div}(a, b) =
			      \begin{cases}
				      \left\lfloor \frac{a}{b} \right\rfloor, & \text{if } \frac{a}{b} \geq 0, \\
				      \left\lceil \frac{a}{b} \right\rceil,   & \text{if } \frac{a}{b} < 0,
			      \end{cases}
		      \]
		      where \(\left\lfloor x \right\rfloor\) is the floor function returning the greatest integer less than or equal to \(x\), and \(\left\lceil x \right\rceil\) is the ceiling function returning the smallest integer greater than or equal to \(x\).

		      This choice ensures that the quotient \(\operatorname{div}(a,b)\) is selected to guarantee a non-negative remainder in the corresponding modulo operation.

		\item The modulo function, denoted \(\operatorname{mod} : \mathbb{Z} \times (\mathbb{Z} \setminus \{0\}) \to \mathbb{Z}\), is defined by
		      \[
			      \operatorname{mod}(a, b) = a - b \cdot \operatorname{div}(a, b),
		      \]
		      which yields the unique remainder satisfying
		      \[
			      0 \leq \operatorname{mod}(a,b) < |b|.
		      \]
	\end{enumerate}

	\bigskip

	\textbf{Rounding direction of \(\operatorname{div}(a,b)\) depends on signs of \(a\) and \(b\) as summarized in the following table:}

	\[
		\begin{array}{|c|c|c|c|c|}
			\hline
			\text{Case} & a & b & \frac{a}{b} & \operatorname{div}(a,b) \text{ rounding} \\
			\hline
			1           & + & + & +           & \text{floor}                             \\
			2           & + & - & -           & \text{ceiling}                           \\
			3           & - & + & -           & \text{ceiling}                           \\
			4           & - & - & +           & \text{floor}                             \\
			\hline
		\end{array}
	\]

	\bigskip

	Here, "\text{floor}" means \(\left\lfloor \frac{a}{b} \right\rfloor\) and "\text{ceiling}" means \(\left\lceil \frac{a}{b} \right\rceil\).

	\bigskip
}
These definitions ensure that for any integers \(a, b\) with \(b \neq 0\), the division algorithm holds:
\[
	a = b \cdot \operatorname{div}(a, b) + \operatorname{mod}(a, b),
\]
with the remainder \(\operatorname{mod}(a,b)\) uniquely satisfying
\[
	0 \leq \operatorname{mod}(a,b) < |b|.
\]
Let's try to prove them!


\lemp{Range of Mod Operation}{
For any integers $a$ and $b$ with $b \neq 0$, the modulo operation satisfies:
\[
	\operatorname{mod}(a, b) \in [0, |b|),
\]
This means that the result of the modulo operation is always a non-negative integer less than the absolute value of $b$.
}{
According to the Division Algorithm, for any integer $a$ and non-zero integer $b$, there exist unique integers $q$ and $r$ such that:
\[
	a = bq + r,
\]
and the remainder satisfies
\[
	0 \leq r < |b|.
\]
Here, we let
\[
	q = \operatorname{div}(a, b), \quad r = \operatorname{mod}(a, b).
\]
This means that the value of the modulo operation $\operatorname{mod}(a,b)$ is the remainder $r$, satisfying the above inequality.
By definition, we have
\[
	\operatorname{mod}(a, b) = a - b \cdot \operatorname{div}(a, b).
\]
To ensure $0 \leq \operatorname{mod}(a,b) < |b|$, we need to choose a suitable quotient so that the remainder falls within this interval. In the definition based on the floor function, when $b>0$, selecting $\operatorname{div}(a,b) = \left\lfloor \frac{a}{b} \right\rfloor$ can meet the requirement. But when $b < 0$, $\left\lfloor \frac{a}{b} \right\rfloor$ may not satisfy this condition.
Therefore, to ensure that the remainder is non-negative and less than $|b|$, when $b < 0$, we should define
\[
	\operatorname{div}(a,b) = \left\lceil \frac{a}{b} \right\rceil,
\]
such that
\[
	0 \leq a - b \cdot \operatorname{div}(a,b) < |b|.
\]
In summary, regardless of whether $b$ is positive or negative, there always exists a suitable integer quotient $\operatorname{div}(a,b)$ such that the remainder $\operatorname{mod}(a,b)$ satisfies
\[
	0 \leq \operatorname{mod}(a,b) < |b|.
\]
This completes the proof.
}

\prob{
	Prove that for any integers $a$ and $b$ with $b \neq 0$,
	$$
		a = b \cdot \operatorname{div}(a, b) + \operatorname{mod}(a, b).
	$$
}


\prob{
	Use the $\operatorname{div}$ and $\operatorname{mod}$ functions from the division algorithm to define the following three Boolean functions:

	\textbf{AND Function}: Define the AND operation for Boolean variables $x, y \in \{0,1\}$, i.e.,
	\[
		\operatorname{and}(x,y) = ?
	\]

	\textbf{OR Function}: Define the OR operation for Boolean variables $x, y \in \{0,1\}$, i.e.,
	\[
		\operatorname{or}(x,y) = ?
	\]

	\textbf{XOR Function}: Define the XOR operation for Boolean variables $x, y \in \{0,1\}$, i.e.,
	\[
		\operatorname{xor}(x,y) = ?
	\]

	Please define these functions using expressions containing only $\operatorname{div}$ and $\operatorname{mod}$.

}

\subsection{Modular Congruence}
With basic notion of modular arithmetic, we can define the modular congruence. This is one of the most important relations in mathematics, and it is widely used in number theory, cryptography, and computer science.

\subsubsection{Definition of Modular Congruence}
\defn{Modular Congruence}{
	Let $a, b \in \mathbb{Z}$ and $m \in \mathbb{N}$ with $m > 0$. We say that $a$ is \textbf{congruent} to $b$ \textbf{modulo} $m$ if $m$ divides the difference $a - b$; that is, $m \mid (a-b)$, or $a - b = k \cdot m, k \in \mathbb{Z}$. This relationship is denoted by
	\[
		a \equiv b \pmod{m}.
	\]
	Or equivalently, we can say that $a$ and $b$ are congruent modulo $m$ if they have the same remainder when divided by $m$. In this case, we can also write:
	\[
		a \mod m = b \mod m.
	\]

	We call $a \equiv b \pmod{m}$ a \textbf{congruence}, and $m$ is called the \textbf{modulus} (plural: moduli). If $a$ and $b$ are not congruent modulo $m$, we write $a \not\equiv b \pmod{m}$.
}

\prob{
	Prove that $a \equiv b \pmod{m}$, if an only if $a \mod m = b \mod m$.
}


\subsubsection{Properties of Modular Congruence}
\lem{}{
	Let $m$ be a positive integer. The integers $a$ and $b$ are congruent modulo $m$ if and only if there is an integer $k$ such that $a = b + km$.
}

\thm{Linearity of Congruences}{
	Let $m$ be a positive integer. If $a \equiv b \pmod{m}$ and $c \equiv d \pmod{m}$, then the following properties hold:
	\[
		a \pm c \equiv b \pm d \pmod{m},
	\]

	\[
		a \cdot c \equiv b \cdot d \pmod{m}.
	\]
}


\thm{Modular Arithmetic}{
	Let \( a, b, c \in \mathbb{Z} \) and \( m \in \mathbb{N} \) with \( m > 0 \). Then the following properties hold:
	\begin{enumerate}
		\item If \( a \equiv b \pmod{m} \), then \( a + c \equiv b + c \pmod{m} \).
		\item If \( a \equiv b \pmod{m} \), then \( a - c \equiv b - c \pmod{m} \).
		\item If \( a \equiv b \pmod{m} \), then \( a \cdot c \equiv b \cdot c \pmod{m} \).
	\end{enumerate}
}

\rmkb{
	These two theorems are very useful in modular arithmetic, and on top of that, we can actually take a glimpse of the algebraic significance of the modular arithmetic. If you know basic algebra, you may find that the modular congruence $m$ is actually an equivalence relation, and the set of integers modulo $m$ forms a ring. This explains why we can copy what we do in basic arithmetic on integers to modular arithmetic.
}

\subsubsection{Modular Congruence as Relation}
\clmp{Modular Congruence is Equivalence}{
	Let $a, b, c \in \mathbb{Z}$ and $m \in \mathbb{N}$ with $m > 0$. The relation \( a \equiv b \pmod{m} \) is an equivalence relation on the set of integers. Specifically, it satisfies the following properties:
	\begin{enumerate}
		\item \textbf{Reflexivity}: For any integer \( a \), \( a \equiv a \pmod{m} \).
		\item \textbf{Symmetry}: If \( a \equiv b \pmod{m} \), then \( b \equiv a \pmod{m} \).
		\item \textbf{Transitivity}: If \( a \equiv b \pmod{m} \) and \( b \equiv c \pmod{m} \), then \( a \equiv c \pmod{m} \).
	\end{enumerate}
}{
	To show that \( a \equiv b \pmod{m} \) is an equivalence relation, we need to prove the three properties.

	\begin{enumerate}
		\item \textbf{Reflexivity}: For any integer \( a \), we have \( a - a = 0 \), and since \( m \mid 0 \), it follows that \( a \equiv a \pmod{m} \).

		\item \textbf{Symmetry}: If \( a \equiv b \pmod{m} \), then by definition, there exists an integer \( k \) such that \( a - b = k m \). Rearranging gives \( b - a = -k m \), which shows that \( m \mid (b - a) \). Thus, \( b \equiv a \pmod{m} \).
		\item \textbf{Transitivity}: If \( a \equiv b \pmod{m} \) and \( b \equiv c \pmod{m} \), then there exist integers \( k_1 \) and \( k_2 \) such that:
		      \[
			      a - b = k_1 m,
		      \]
		      and
		      \[
			      b - c = k_2 m.
		      \]
		      Adding these two equations gives:
		      \[
			      a - c = (a - b) + (b - c) = (k_1 + k_2) m.
		      \]
		      This shows that \( m \mid (a - c) \), which implies \( a \equiv c \pmod{m} \).
	\end{enumerate}
	Thus, we have shown that \( a \equiv b \pmod{m} \) is an equivalence relation.
}
\bigskip

We can define a \textbf{congruence class} as the set of all integers that are congruent to a given integer \( a \) modulo \( m \). These equivalence classes are very useful in mathematics, and thus mathematicians give them a special name: \textbf{residue classes} (or congruence class), as a special case of equivalence classes.
To recap on the definition of equivalence classes:
\defn{Equivalence Class}{
	Let \( R \) be an equivalence relation on a set \( S \). For any element \( a \in S \), the equivalence class of \( a \) under the relation \( R \) is defined as:
	\[
		[a] = \{ x \in S : x R a \}.
	\]
	The set of all equivalence classes forms a partition of the set \( S \).
}

Substituting the relation in previous definition with modular congruence $m$, we have:
\defn{Residue/Congruence Class}{
	Let \( m \in \mathbb{N} \) with \( m > 0 \). The residue class of an integer \( a \) modulo \( m \) is defined as:
	\[
		[a]_m = \{ x \in \mathbb{Z} : x \equiv a \pmod{m} \}.
	\]
	The set of all residue classes modulo \( m \) forms a partition of the integers \( \mathbb{Z} \).
}

\ex{
	For example, the residue classes modulo 3 are:
	\[
		[0]_3 = \{ \ldots, -6, -3, 0, 3, 6, \ldots \},
	\]
	\[
		[1]_3 = \{ \ldots, -5, -2, 1, 4, 7, \ldots \},
	\]
	and
	\[
		[2]_3 = \{ \ldots, -4, -1, 2, 5, 8, \ldots \}.
	\]
}

\prob{
	Let \( m \) be a non-zero integer. Let \( \equiv \pmod m \) be the congruence relation defined on \( \mathbb{Z} \) by \( a \equiv b \pmod m \) if and only if \( m \) divides \( a-b \). Prove that the equivalence classes of \( \mathbb{Z} \) under this relation, namely the congruence classes modulo \( m \), form a partition of \( \mathbb{Z} \).
}


\subsection{Arithmetic Modulo $m$}
Now we can formally define the arithmetic operations on the residue classes modulo \( m \). The operations are defined as follows:
\defn{Arithmetic Modulo $m$}{
	Let \( m \in \mathbb{N} \) with \( m > 0 \). The arithmetic operations on the residue classes modulo \( m \) are defined as follows:
	\begin{enumerate}
		\item Addition: For any \( [a]_m, [b]_m \in \mathcal{C} \), define
		      \[
			      [a]_m + [b]_m = [a + b]_m.
		      \]
		\item Subtraction: For any \( [a]_m, [b]_m \in \mathcal{C} \), define
		      \[
			      [a]_m - [b]_m = [a - b]_m.
		      \]
		\item Multiplication: For any \( [a]_m, [b]_m \in \mathcal{C} \), define
		      \[
			      [a]_m \cdot [b]_m = [a \cdot b]_m.
		      \]
	\end{enumerate}
}
\rmkb{
Note that the operations are well-defined, meaning that the result does not depend on the choice of representatives from the equivalence classes. This is because if \( [a]_m = [a']_m \) and \( [b]_m = [b']_m \), then:
\[
	a - a' \equiv 0 \pmod{m} \quad \text{and} \quad b - b' \equiv 0 \pmod{m}.
\]
Hence,
\[
	[a + b]_m = [a' + b']_m,
\]
and similarly for subtraction and multiplication.
}

All the algebraic properties of addition and multiplication hold in the residue classes modulo \( m \). Specifically, the following properties hold:

Let \( m \in \mathbb{N} \) with \( m > 0 \), $\mathcal{C} = \{ [0]_m, [1]_m, \dots, [m-1]_m \}$ be the set of residue classes modulo \( m \). The arithmetic operations on the residue classes modulo \( m \) satisfy the following properties:
\begin{enumerate}
	\item Closure: For any \( [a]_m, [b]_m \in \mathcal{C} \), both \( [a]_m + [b]_m \) and \( [a]_m \cdot [b]_m \) are also in \( \mathcal{C} \).
	\item Associativity: For any \( [a]_m, [b]_m, [c]_m \in \mathcal{C} \),
	      \[
		      ([a]_m + [b]_m) + [c]_m = [a]_m + ([b]_m + [c]_m),
	      \]
	      and
	      \[
		      ([a]_m \cdot [b]_m) \cdot [c]_m = [a]_m \cdot ([b]_m \cdot [c]_m).
	      \]
	\item Commutativity: For any \( [a]_m, [b]_m \in \mathcal{C} \),
	      \[
		      [a]_m + [b]_m = [b]_m + [a]_m,
	      \]
	      and
	      \[
		      [a]_m \cdot [b]_m = [b]_m \cdot [a]_m.
	      \]
	\item Distributivity: For any \( [a]_m, [b]_m, [c]_m \in \mathcal{C} \),
	      \[
		      [a]_m([b]_m + [c]_m) = ([a]_m[b]_m + [a]_m[c]_m).
	      \]
\end{enumerate}
You may try to prove these properties as an exercise. The proof is similar to the proof of the properties of addition and multiplication in the integers, but we need to be careful about the equivalence classes.

\defn{Well-defined Operation}{
Let $S$ be a set whose elements are equivalence classes of some underlying set $X$ with an equivalence relation $\sim$.
We say a binary operation $*$ on $S$ is \textbf{well-defined} if for any $[x], [y] \in S$ (where $[x]$ and $[y]$ denote the equivalence classes containing $x$ and $y$ respectively), the result $[x] * [y]$ depends \emph{only} on the equivalence classes $[x]$ and $[y]$ themselves, and \emph{not} on the particular choice of representatives used to define or compute it.

More formally, suppose the operation $*$ on $S$ is defined by first choosing representatives $x' \in [x]$ and $y' \in [y]$ and then using an operation $\star$ from the underlying set $X$ (or related to $X$), such that $[x] * [y] := [x' \star y']$. For this definition to be well-defined, it must be true that if $x_1, x_2$ are any two representatives of $[x]$ (i.e., $[x_1] = [x_2]$) and $y_1, y_2$ are any two representatives of $[y]$ (i.e., $[y_1] = [y_2]$), then the result obtained using $x_1, y_1$ must be the same as the result obtained using $x_2, y_2$:
\[
	[x_1 \star y_1] = [x_2 \star y_2]
\]
}

\ex{
For example, the addition and multiplication operations on the residue classes modulo \( m \) are well-defined. This means that if \( [a]_m = [a']_m \) and \( [b]_m = [b']_m \), then:

\[
	[a]_m + [b]_m = [a' + b']_m,
\]

and

\[
	[a]_m \cdot [b]_m = [a' \cdot b']_m.
\]
}

\section{Algebraic Structures}
This section introduces some basic ideas in algebraic structures, which are fundamental concepts in abstract algebra. We will discuss groups, rings, and fields, which are important structures in mathematics, with modular congruence as a motivating example.
\subsection{Quotient Sets}
\defn{Quotient Set}{
	Let \( S \) be a set and \( R \) be an equivalence relation on \( S \). The quotient set \( S/R \) is the set of all equivalence classes of \( S \) under the relation \( R \). It is denoted as:
	\[
		S/R = \{ [x] : x \in S \}.
	\]
}
\ex{
	Recall in the previous section, we see $\mathcal{C} = \{ [0]_m, [1]_m, \dots, [m-1]_m \}$ forms a set of all equivalence classes of integers modulo \( m \) in the set of integers \( \mathbb{Z} \). This is a quotient set, and we can denote it as \( \mathbb{Z}/m\mathbb{Z} \) or simply \( \mathbb{Z}/m \).
}


\prob{
We say a relation or an operation is \textbf{well-defined} if it is \textbf{only determined by the objects to which it is applied}, and not by the particular representatives of those objects. Prove that the addition and multiplication operations on the residue classes modulo \( m \) on $\mathbb{Z}$ are well-defined.
Namely, show that ($a'$ and $b'$ are representatives of $[a]_m$ and $[b]_m$ respectively):
\[
	[a]_m + [b]_m = [a' + b']_m,
\]

\[
	[a]_m \cdot [b]_m = [a' \cdot b']_m.
\]
Note the operations on the left-hand side are defined sequentially, for example,
\[
	\{1, 2, 3\} + \{4, 5, 6\} = \{5, 7, 9\}.
\]
}

We see that we can extend similar algebraic and arithmetic properties from one set to another different set that share some common nature. This is the core idea of algebraic structures, where we abstract the properties of a set and define new sets with similar properties. The most basic algebraic structures are groups, rings, and fields.

\defn{Algebraic Structure}{
	An algebraic structure consists of a set \( S \) along with one or more operations defined on that set. The operations must satisfy certain properties, which can include closure, associativity, commutativity, and the existence of identity and inverse elements.
}

Mathematicians study these structures to understand the underlying properties and relationships between different sets and operations. Some of them are widely discovered and studied, these include:
\begin{itemize}
	\item \textbf{Groups}: A set with a single binary operation that satisfies closure, associativity, identity, and invertibility.
	\item \textbf{Rings}: A set with two binary operations (addition and multiplication) that satisfy certain properties, including distributivity.
	\item \textbf{Fields}: A set with two binary operations (addition and multiplication) that satisfy all the properties of a ring, along with the existence of multiplicative inverses for non-zero elements.
\end{itemize}
\subsection{Groups}
\defn{Group}{
	A group is a set \( G \) equipped with a binary operation \( * \) that satisfies the following properties:
	\begin{enumerate}
		\item \textbf{Closure}: For all \( a, b \in G \), \( a * b \in G \).
		\item \textbf{Associativity}: For all \( a, b, c \in G \), \( (a * b) * c = a * (b * c) \).
		\item \textbf{Identity Element}: There exists an element \( e \in G \) such that for all \( a \in G \), \( e * a = a * e = a \).
		\item \textbf{Inverse Element}: For each \( a \in G \), there exists an element \( b \in G \) such that \( a * b = b * a = e \).
	\end{enumerate}
	The notation \( (G, *) \) is often used to denote a group, where \( G \) is the set and \( * \) is the operation.
	We say that \( G \) is an \textbf{abelian group} if the operation \( * \) is commutative, i.e., for all \( a, b \in G \), \( a * b = b * a \).
}
\ex{
	\begin{itemize}
		\item The set of integers \( \mathbb{Z} \) under addition is a group. The identity element is 0, and the inverse of any integer \( a \) is \( -a \).
		\item The set of non-zero rational numbers \( \mathbb{Q}^* \) under multiplication is also a group. The identity element is 1, and the inverse of any non-zero rational number \( a \) is \( \frac{1}{a} \).
		\item The addition of integers modulo \( m \) forms a group. The identity element is \( [0]_m \), and the inverse of any element \( [a]_m \) is \( [m - a]_m \).
	\end{itemize}
}

\propp{
	Let $\mathbb{Z}_m$ be the set of integers modulo \( m \) under addition. Prove that $(\mathbb{Z}_m, +)$ is a group.
}{
	To show that \( (\mathbb{Z}_m, +) \) is a group, we need to verify the four properties of a group:
	\begin{enumerate}
		\item \textbf{Closure}: For any \( [a]_m, [b]_m \in \mathbb{Z}_m \), we have:
		      \[
			      [a]_m + [b]_m = [a + b]_m.
		      \]
		      Since \( a + b \) is an integer, \( [a + b]_m \) is also in \( \mathbb{Z}_m \). Thus, closure holds.

		\item \textbf{Associativity}: For any \( [a]_m, [b]_m, [c]_m \in \mathbb{Z}_m \), we have:
		      \[
			      ([a]_m + [b]_m) + [c]_m = [a + b + c]_m = [a]_m + ([b]_m + [c]_m).
		      \]
		      Thus, associativity holds.

		\item \textbf{Identity Element}: The identity element in \( (\mathbb{Z}_m, +) \) is \( [0]_m \). For any \( [a]_m \in \mathbb{Z}_m \):
		      \[
			      [a]_m + [0]_m = [a + 0]_m = [a]_m.
		      \]
		      Thus, the identity element exists.

		\item \textbf{Inverse Element}: For any \( [a]_m \in \mathbb{Z}_m \), the inverse element is \( [-a]_m = [(m - a)]_m \). We have:
		      \[
			      [a]_m + [-a]_m = [(a - a)]_m = [0]_m.
		      \]
		      Thus, the inverse element exists.
	\end{enumerate}
	Since all four properties hold, we conclude that \( (\mathbb{Z}_m, +) \) is a group.
}
\rmkb{
	Or we may just reuse the conclusion that addition on $\mathbb{Z}$ is a group, and the addition on $\mathbb{Z}_m$ is a subset of $\mathbb{Z}$, so it preserves the group structure and properties.

	In a nutshell, if a bigger structure have a certain property, then any subset of it will also have the same property. This is a very common and useful idea in mathematics, and we will see more of it in the future.
}

\prob{
	Let $(G, \cdot)$ be a group such that $x^2 = e$ for all $x \in G$, where $e$ is the identity element of $G$. Prove that $G$ is an Abelian group. That is, prove that $xy = yx$ for all $x, y \in G$.
}

\subsection{Rings}

\defn{Ring}{
	A ring is a set \( R \) equipped with two binary operations \( + \) and \( \cdot \) that satisfy the following properties:
	\begin{enumerate}
		\item \textbf{Additive Closure}: For all \( a, b \in R \), \( a + b \in R \).
		\item \textbf{Additive Associativity}: For all \( a, b, c \in R \), \( (a + b) + c = a + (b + c) \).
		\item \textbf{Additive Identity}: There exists an element \( 0_R \in R \) such that for all \( a \in R \), \( a + 0_R = 0_R + a = a \).
		\item \textbf{Additive Inverse}: For each \( a \in R \), there exists an element \( -a \in R \) such that \( a + (-a) = 0_R \).
		\item \textbf{Multiplicative Closure}: For all \( a, b \in R \), \( a \cdot b \in R \).
		\item \textbf{Multiplicative Associativity}: For all \( a, b, c \in R \), \( (a \cdot b) \cdot c = a \cdot (b \cdot c) \).
		\item \textbf{Distributive Property}: For all \( a, b, c \in R \):
		      - Left Distributive: \( a(b + c) = ab + ac \)
		      - Right Distributive: \( (a + b)c = ac + bc \)
	\end{enumerate}
	A ring is called a commutative ring if the multiplication operation is commutative, i.e., for all \( a, b \in R, ab = ba\). A ring with unity is called an unital ring if it contains an identity element for multiplication.
}
\ex{
	The set of integers \( \mathbb{Z} \) under addition and multiplication is a ring. The additive identity is 0, and the multiplicative identity is 1.
}
\ex{
	The set of polynomials with real coefficients forms a ring under polynomial addition and multiplication. The additive identity is the zero polynomial, and the multiplicative identity is the constant polynomial 1. This is commonly denoted as \( \mathbb{R}[x] \). Polynomial rings are important in algebra and number theory, as they provide a way to study polynomial equations and their roots.
}

\ex{
	The set of \( n \times n \) matrices over \( \mathbb{R} \) forms a ring under matrix addition and multiplication. The additive identity is the zero matrix ($\mathbf{0}$), and the multiplicative identity is the identity matrix ($\mathbf{I}$). This is commonly denoted as \( M_n(\mathbb{R}) \), where \( n \) is the size of the matrices and \( \mathbb{R} \) is the field over which the matrices are defined.
}

Matrices is a very important topic in algebra, and it has some very interesting properties and relations on number theory and algebra. If you you are interested, you may see \href{https://en.wikipedia.org/wiki/Vandermonde_matrix}{Vandermonde Matrix} and \href{https://en.wikipedia.org/wiki/Cayley%E2%80%93Hamilton_theorem}{Caley-Hamilton Theorem} for more details.

\defn{Idempotent Element}{
	An element \( a \) in a ring \( R \) is called idempotent if \( a^2 = a \). In other words, applying the operation to the element with itself yields the same element.
}

\defn{Integral Domain}{
	An integral domain is a commutative ring \( R \) with unity (multiplicative identity) such that:
	\begin{enumerate}
		\item \( R \) has no zero divisors, meaning if \( ab = 0 \) for \( a, b \in R \), then either \( a = 0 \) or \( b = 0 \).
		\item The cancellation property holds: If \( a, b, c \in R \) and \( ab = ac \), then \( b = c \) if \( a \neq 0 \).
	\end{enumerate}
	An integral domain is a special type of ring that allows for division (except by zero) and has a well-defined notion of multiplication.
}

\ex{
	The set of integers \( \mathbb{Z} \) is an integral domain. The product of two non-zero integers is non-zero, and the cancellation property holds.
}
\ex{
	The set of Real numbers \( \mathbb{R} \) is also an integral domain. The product of two non-zero real numbers is non-zero, and the cancellation property holds.
}
\ex{
	The set of polynomials with real coefficients \( \mathbb{R}[x] \) is an integral domain. The product of two non-zero polynomials is non-zero, and the cancellation property holds.
}
\ex{
	The set of \( n \times n \) matrices over \( \mathbb{R} \) is not an integral domain because it contains zero divisors. For example, the product of two non-zero matrices can be the zero matrix.
}
\ex{
	The set of complex numbers \( \mathbb{C} \) is an integral domain. The product of two non-zero complex numbers is non-zero, and the cancellation property holds.
}
\prob{
	The set of real functions \(\mathbb{R}^\mathbb{R}\) is an integral domain. The product of two non-zero functions is non-zero, and the cancellation property holds. Is this true?
}

\subsection{Polynomial Rings}

\defn{Polynomial Ring $R[x]$}{
	Let $R$ be a ring. The polynomial ring $R[x]$ with coefficients in $R$ is defined as the set of all infinite sequences $(a_0, a_1, a_2, \dots)$ satisfying the following conditions:
	\begin{enumerate}
		\item Each element $a_i$ belongs to the ring $R$ ($a_i \in R$ for all $i \ge 0$).
		\item Only a finite number of $a_i$ are non-zero. That is, there exists a non-negative integer $N$ (which depends on the sequence) such that for all $i > N$, $a_i = 0$.
	\end{enumerate}
	We define addition and multiplication operations on this set $R[x]$ as follows:

	Let $p = (a_0, a_1, a_2, \dots)$ and $q = (b_0, b_1, b_2, \dots)$ be any two elements in $R[x]$ (i.e., polynomials).

	\paragraph{Addition ($+$):} The sum $p+q$ is a new sequence $(c_0, c_1, c_2, \dots)$, where the $k$-th element $c_k$ is defined as the sum of $a_k$ and $b_k$ in the ring $R$:
	$$c_k = a_k +_R b_k$$
	Formally: $p+q = (a_0+b_0, a_1+b_1, a_2+b_2, \dots)$, where all additions $a_i+b_i$ are performed in the ring $R$.

	\paragraph{Multiplication ($\cdot$):} The product $p \cdot q$ is a new sequence $(d_0, d_1, d_2, \dots)$, where the $k$-th element $d_k$ is defined by the following sum involving multiplication and addition in the ring $R$:
	$$d_k = \sum_{i=0}^k a_i \cdot_R b_{k-i} = a_0 \cdot b_k +_R a_1 \cdot b_{k-1} +_R \dots +_R a_k \cdot b_0$$
	Formally: $p \cdot q = (d_0, d_1, d_2, \dots)$, where $d_k = \sum_{i=0}^k a_i b_{k-i}$. This sum is known as the \textbf{Cauchy product} or \textbf{Convolution}.

	With these operations, the set $R[x]$ forms a ring. The zero element is the sequence $(0, 0, 0, \dots)$, and if $R$ has a multiplicative identity $1_R$, the multiplicative identity of $R[x]$ is the sequence $(1_R, 0, 0, \dots)$.
}

\paragraph{Correspondence to Standard Notation:}
The formal definition using sequences corresponds directly to the standard polynomial notation $\sum_{i=0}^n a_i x^i$.
\begin{itemize}
	\item The sequence $(a, 0, 0, \dots)$ corresponds to the constant polynomial $a \in R$.
	\item The sequence $(0, 1_R, 0, 0, \dots)$ corresponds to the indeterminate $x$.
	\item The sequence $(0, \dots, 0, 1_R \text{ at position } k, 0, \dots)$ corresponds to $x^k$.
	\item The polynomial $a_n x^n + \dots + a_1 x + a_0$ corresponds to the sequence $(a_0, a_1, \dots, a_n, 0, 0, \dots)$.
\end{itemize}
The addition and multiplication rules for these sequences precisely mirror the standard rules for polynomial addition (adding like terms) and multiplication (expanding and combining like terms, which results in the convolution formula for coefficients). Thus, the familiar notation $\sum a_i x^i$ is a convenient representation for these finite sequences under the defined operations.

\rmkb{
	Polynomial rings can be related to series and generating functions. You may explore these topics further if you are interested.
}

\propp{
	If a ring \( R \) is an integral domain, then the polynomial ring \( R[x] \) is also an integral domain.
}{
	To prove that $R[x]$ is an integral domain, we need to show that it satisfies the three conditions for an integral domain:
	\begin{enumerate}
		\item $R[x]$ is a commutative ring.
		\item $R[x]$ has a unity element $1_{R[x]} \neq 0_{R[x]}$.
		\item $R[x]$ has no zero divisors.
	\end{enumerate}
	\textbf{1. $R[x]$ is a commutative ring:}

	The set $R[x]$ consists of polynomials with coefficients in $R$. Polynomial addition and multiplication are defined as follows:
	For $p(x) = \sum_{i=0}^n a_i x^i$ and $q(x) = \sum_{j=0}^m b_j x^j$,
	\begin{align*} (p+q)(x) &= \sum_{k=0}^{\max(n,m)} (a_k +_R b_k) x^k \quad (\text{where } a_k=0 \text{ for } k>n, b_k=0 \text{ for } k>m) \\ (p \cdot q)(x) &= \sum_{k=0}^{n+m} c_k x^k, \quad \text{where } c_k = \sum_{i=0}^k a_i \cdot_R b_{k-i} \end{align*}
	It is a standard result that $(R[x], +, \cdot)$ forms a ring. We need to verify commutativity of multiplication.
	Let $p(x) = \sum a_i x^i$ and $q(x) = \sum b_j x^j$.
	The $k$-th coefficient of $p(x)q(x)$ is $c_k = \sum_{i=0}^k a_i b_{k-i}$.
	The $k$-th coefficient of $q(x)p(x)$ is $d_k = \sum_{j=0}^k b_j a_{k-j}$.
	Let $i' = k-j$, so $j = k-i'$. As $j$ ranges from $0$ to $k$, $i'$ ranges from $k$ to $0$. Thus, $d_k = \sum_{i'=0}^k b_{k-i'} a_{i'}$.
	Since $R$ is a commutative ring (as it is an integral domain), $a_{i'} b_{k-i'} = b_{k-i'} a_{i'}$.
	Therefore, $d_k = \sum_{i'=0}^k a_{i'} b_{k-i'} = c_k$.
	Since the coefficients are the same for all $k$, $p(x)q(x) = q(x)p(x)$. Thus, multiplication in $R[x]$ is commutative.

	\textbf{2. $R[x]$ has a unity element $1_{R[x]} \neq 0_{R[x]}$:}

	Since $R$ is an integral domain, it has a multiplicative identity $1_R \neq 0_R$.
	Consider the constant polynomial $1_{R[x]} = 1_R \cdot x^0 = (1_R, 0, 0, \dots)$.
	Let $p(x) = \sum_{i=0}^n a_i x^i = (a_0, a_1, \dots, a_n, 0, \dots)$.
	The product $p(x) \cdot 1_{R[x]}$ has its $k$-th coefficient as $\sum_{i=0}^k a_i (1_{R[x]})_{k-i}$. The only non-zero coefficient of $1_{R[x]}$ is $(1_{R[x]})_0 = 1_R$. So the sum is non-zero only when $k-i=0$, i.e., $i=k$. The term is $a_k \cdot (1_{R[x]})_0 = a_k \cdot 1_R = a_k$. Thus, $p(x) \cdot 1_{R[x]} = p(x)$.
	Similarly, $1_{R[x]} \cdot p(x) = p(x)$.
	So, $1_{R[x]}$ is the multiplicative identity in $R[x]$.
	The zero element in $R[x]$ is the zero polynomial $0_{R[x]} = (0_R, 0_R, \dots)$. Since $1_R \neq 0_R$ in the integral domain $R$, the first coefficient of $1_{R[x]}$ is different from the first coefficient of $0_{R[x]}$. Therefore, $1_{R[x]} \neq 0_{R[x]}$.

	\textbf{3. $R[x]$ has no zero divisors:}

	We need to show that if $p(x), q(x) \in R[x]$ are non-zero polynomials, then their product $p(x)q(x)$ is also non-zero.
	Let $p(x) = a_n x^n + a_{n-1} x^{n-1} + \dots + a_0$, where $a_n \neq 0_R$ (so $\deg(p) = n$).
	Let $q(x) = b_m x^m + b_{m-1} x^{m-1} + \dots + b_0$, where $b_m \neq 0_R$ (so $\deg(q) = m$).
	Consider the product $p(x)q(x) = c_{n+m} x^{n+m} + \dots + c_1 x + c_0$.
	The coefficient of the highest degree term, $x^{n+m}$, is given by:
	$$ c_{n+m} = \sum_{i=0}^{n+m} a_i b_{n+m-i} $$
	In this sum, if $i > n$, then $a_i = 0_R$. If $n+m-i > m$ (which means $n > i$), then $b_{n+m-i} = 0_R$.
	For a term $a_i b_{n+m-i}$ to be potentially non-zero, we must have $i \le n$ and $n+m-i \le m$ (which implies $n \le i$).
	The only integer $i$ satisfying both $i \le n$ and $n \le i$ is $i=n$.
	Therefore, the sum reduces to a single term:
	$$ c_{n+m} = a_n b_{n+m-n} = a_n b_m $$
	Since $R$ is an integral domain and $a_n \neq 0_R$ and $b_m \neq 0_R$, their product $a_n b_m \neq 0_R$.
	Thus, the coefficient $c_{n+m}$ of the $x^{n+m}$ term in $p(x)q(x)$ is non-zero.
	This implies that the product polynomial $p(x)q(x)$ is not the zero polynomial ($0_{R[x]}$).
	Therefore, $R[x]$ has no zero divisors.
	\textbf{Conclusion:}
	Since $R[x]$ is a commutative ring with a non-zero unity element and has no zero divisors, $R[x]$ is an integral domain.
}
\subsection{Fields}
\defn{Field}{
	A \textbf{field} is a set \( F \) equipped with two binary operations, addition \((+)\) and multiplication \((\cdot)\), satisfying the following axioms:

	\begin{enumerate}
		\item \textbf{Additive Structure:}
		      \begin{enumerate}
			      \item (\textit{Closure}) For all \( a, b \in F \), \( a + b \in F \).
			      \item (\textit{Associativity}) For all \( a, b, c \in F \), \( (a + b) + c = a + (b + c) \).
			      \item (\textit{Identity}) There exists \( 0 \in F \) such that \( a + 0 = 0 + a = a \) for all \( a \in F \).
			      \item (\textit{Inverse}) For each \( a \in F \), there exists \( -a \in F \) such that \( a + (-a) = 0 \).
			      \item (\textit{Commutativity}) For all \( a, b \in F \), \( a + b = b + a \).
		      \end{enumerate}
		\item \textbf{Multiplicative Structure:}
		      \begin{enumerate}
			      \item (\textit{Closure}) For all \( a, b \in F \), \( a \cdot b \in F \).
			      \item (\textit{Associativity}) For all \( a, b, c \in F \), \( (a \cdot b) \cdot c = a \cdot (b \cdot c) \).
			      \item (\textit{Identity}) There exists \( 1 \in F \), \( 1 \neq 0 \), such that \( a \cdot 1 = 1 \cdot a = a \) for all \( a \in F \).
			      \item (\textit{Inverse}) For each \( a \in F \), \( a \neq 0 \), there exists \( a^{-1} \in F \) such that \( a \cdot a^{-1} = 1 \).
			      \item (\textit{Commutativity}) For all \( a, b \in F \), \( a \cdot b = b \cdot a \).
		      \end{enumerate}
		\item \textbf{Distributive Law:} For all \( a, b, c \in F \),
		      \[
			      a \cdot (b + c) = a \cdot b + a \cdot c.
		      \]
	\end{enumerate}

	In summary, a field is a commutative ring with unity in which every nonzero element has a multiplicative inverse. This means that addition, subtraction, multiplication, and division (except by zero) are all well-defined in \( F \).
}

A special field is known as a \textbf{finite field}, which is a field with a finite number of elements. Finite fields are important in number theory, coding theory, and cryptography. All arithmetic operations are defined on finite fields in undergraduate-level mathematics.

\ex{
	The set of real numbers \( \mathbb{R} \) is a field. The operations of addition and multiplication satisfy all the field axioms. The additive identity is 0, the multiplicative identity is 1, and every non-zero real number has a multiplicative inverse. Same goes for the set of complex numbers \( \mathbb{C} \), set of rational numbers \( \mathbb{Q} \), and the set of finite fields \( \mathbb{F}_p \) for prime \( p \).
}

\prob{
	Prove that the set of natural numbers \( \mathbb{N} \) is not a field. You may use known results from number theory and algebra.
}

Another problem on reasoning operations in a field.

\prob{
	Let \( F \) be a field consisting of exactly three elements \( 0, 1, x \). Prove that \( x + x = 1 \) and that \( x \cdot x = 1 \). Obtain the addition and multiplication tables for \( F \).
}

If you know what is a vector space, you may also think of the relation between a field and a vector space. You may check \href{https://math.arizona.edu/~cais/223Page/hout/236w06fields.pdf}{this} out. But we will not go into the details of vector spaces in this lecture.

\chapter{Primes, GCD, LCM}

\section{Prime Numbers}

\defn{Prime Number}{
	An integer $p$ greater than 1 is called prime if the only positive factors of $p$ are 1 and $p$.
	A positive integer that is greater than 1 and is not prime is called \textbf{composite}.
}

The reason why prime numbers are so fascinating to mathematicians is that they have so many interesting and unique property, even except for what we have seen in its definition. Below is what we call the fundamental theorem of arithmetic.

\thm{Fundamental Theorem of Arithmetic}{
For every integer $n > 1$, there exists a unique factorization into prime numbers, up to the order of the factors. Specifically, $n$ can be expressed as
\[ n = p_1^{a_1} \cdot p_2^{a_2} \cdot \ldots \cdot p_k^{a_k} \]
where $p_1 < p_2 < \ldots < p_k$ are prime numbers and $a_1, a_2, \ldots, a_k$ are positive integers. This factorization is unique, apart from the order of the prime factors.
}
\pf{
We prove the theorem in two parts: existence and uniqueness.

\textbf{Existence:} We prove by mathematical induction that every integer greater than 1 can be written as a product of primes.

\textit{Base Case:} For $n = 2$, the statement holds true since 2 is itself a prime number.

\textit{Inductive Step:} Assume the statement holds for all integers greater than 1 and less than $n$. Now consider the integer $n$.
\begin{itemize}
	\item If $n$ is prime, then it is trivially a product of primes (itself).
	\item If $n$ is not prime, it can be written as $n = a \cdot b$ where $1 < a, b < n$. By the inductive hypothesis, both $a$ and $b$ can be factored into a product of primes. Therefore, $n$ can also be expressed as a product of primes by combining the prime factorization of $a$ and $b$.
\end{itemize}
This completes the proof of existence.

\textbf{Uniqueness:} Assume, for the sake of contradiction, that there are two distinct prime factorization of $n$:
\[ n = p_1^{a_1} \cdot p_2^{a_2} \cdot \ldots \cdot p_k^{a_k} = q_1^{b_1} \cdot q_2^{b_2} \cdot \ldots \cdot q_m^{b_m} \]
where $p_i$ and $q_j$ are prime numbers, and $a_i, b_j$ are positive integers.
To proceed to the rest of the proof, we need to use Euclid's lemma. (Just take it as a known result for now, we will revisit this in Bezout's identity.)

\lemp{Euclid's Lemma}{
	Let \(p\) be a prime number. If \(p\) divides the product \(ab\), where \(a\) and \(b\) are integers, then \(p\) divides \(a\) or \(p\) divides \(b\).}{ Assume \(p\) is a prime that divides \(ab\) but does not divide \(a\). We need to show that \(p\) must divide \(b\).

	Since \(p\) does not divide \(a\), the greatest common divisor (gcd) of \(a\) and \(p\) is 1, i.e., \(\text{gcd}(a, p) = 1\). According to Bezout's identity, there exist integers \(x\) and \(y\) such that:

	\[ax + py = 1\]

	Multiplying both sides of the equation by \(b\), we get:

	\[abx + pby = b\]

	Since \(p\) divides \(ab\) (by assumption), \(p\) divides \(abx\). Also, \(p\) obviously divides \(pby\). Hence, \(p\) divides the sum \(abx + pby\), which means \(p\) divides \(b\).

	This completes the proof, showing that if \(p\) divides \(ab\) and does not divide \(a\), then \(p\) must divide \(b\), in accordance with Euclid's lemma.}

By Euclid's lemma, if a prime divides the product of two numbers, it must divide at least one of those numbers. Thus, $p_1$ must divide some $q_j$ on the right-hand side. Since $q_j$ is prime, we conclude that $p_1 = q_j$. Applying this argument symmetrically and repeatedly, we find that each set of prime factors must be identical to the other, contradicting the assumption of two distinct factorization.

Therefore, the prime factorization of any integer greater than 1 is unique, up to the order of the factors, which completes the proof of the Fundamental Theorem of Arithmetic.
}

\ex{
	100 can be taken as $100 = 2\cdot 2\cdot 5\cdot 5 = 2^2\cdot 5^2$.
	Both 2 and 5 are primes.
}

\subsection{Prime Factorization Algorithm}
We can use an iterative algorithm to find the prime factorization of a number. The algorithm works by dividing the number by the smallest prime factor repeatedly until the number becomes 1. The prime factors are collected in a list.
The algorithm can be implemented as follows:
\begin{algorithm}
	\caption{Prime Factorization}
	\begin{algorithmic}[H]
		\Function{PrimeFactorization}{$n$}
		\State $factors \gets \text{an empty list}$
		\For{$p \gets 2$ \textbf{to} $\infty$} \Comment{Iterate over all primes}
		\While{$n \mod p == 0$}
		\State Add $p$ to $factors$
		\State $n \gets n / p$
		\EndWhile
		\If{$p > n / p$}
		\State \textbf{break}
		\EndIf
		\EndFor
		\If{$n > 1$}
		\State Add $n$ to $factors$
		\EndIf
		\State \Return $factors$
		\EndFunction
	\end{algorithmic}
\end{algorithm}
\ex{
	To find the prime factorization of 8964:
	\begin{itemize}
		\item Start with \(n = 8964\).
		\item Divide by \(2\): \(8964 \div 2 = 4482\) (add \(2\) to factors).
		\item Divide by \(2\): \(4482 \div 2 = 2241\) (add \(2\) to factors).
		\item Now, \(2241\) is not divisible by \(2\), so we try the next prime, which is \(3\).
		\item Divide by \(3\): \(2241 \div 3 = 747\) (add \(3\) to factors).
		\item Divide by \(3\): \(747 \div 3 = 249\) (add \(3\) to factors).
		\item Divide by \(3\): \(249 \div 3 = 83\) (add \(3\) to factors).
		\item Now, \(83\) is a prime number.
		\item The final prime factorization is: \(8964 = 2^2 \cdot 3^3 \cdot 83^1\).
	\end{itemize}
}

\subsubsection{Complexity of naive Prime Factorization}
The time complexity of this naive prime factorization algorithm is primarily determined by the outer loop that iterates through potential prime factors $p$. The inner \texttt{while} loop divides the number $n$ by $p$ repeatedly.

Let the initial number be $N$. The outer loop starts with $p=2$ and increments $p$ by 1 in each iteration. The loop terminates when $p > n/p$, which is equivalent to $p^2 > n$. Since $n$ is always decreasing (or stays the same) inside the \texttt{while} loop, the value of $p$ can reach up to $\approx \sqrt{N}$ in the worst case (specifically, if the largest prime factor is close to $\sqrt{N}$ or $N$ itself). The total number of iterations of the outer loop is thus roughly proportional to $\sqrt{N}$.

Inside the \texttt{while} loop, the number $n$ is divided by $p$. For a given prime factor $p$, the \texttt{while} loop runs $v_p(N)$ times, where $v_p(N)$ is the exponent of $p$ in the prime factorization of $N$. The total number of divisions performed across all prime factors is $O(\log N)$, since each division reduces the magnitude of $n$ significantly, and the product of prime factors is $N$.

However, the dominant cost comes from the outer loop's progression and the modulo operations. For each value of $p$ in the outer loop (up to $\approx \sqrt{N}$), we perform a modulo check $n \mod p == 0$ at least once. The total number of such checks is $O(\sqrt{N})$. Although the \texttt{while} loop divisions are efficient in total, the number of attempts to divide by $p$ in the outer loop dominates the runtime.

Therefore, the time complexity of this naive algorithm is $O(\sqrt{N})$.

The space complexity is determined by the size of the list storing the prime factors. The number of prime factors of $N$ (counting multiplicity) is at most $O(\log N)$ (e.g., for $N = 2^k$, there are $k = \log_2 N$ factors of 2). Thus, the space complexity is $O(\log N)$.

\subsubsection{Correctness of the Algorithm}
To demonstrate the correctness of the algorithm, we need to show that it finds all prime factors of the input number $n$ with their correct multiplicities.

1.  \textbf{Finding the smallest prime factor:} The algorithm iterates through integers $p = 2, 3, 4, \ldots$ in increasing order. When it finds a number $p$ that divides the current value of $n$, this $p$ must be the smallest prime factor of the current $n$. Why? Because if $n$ had a smaller prime factor $q < p$, then $q$ would have been checked earlier in the outer loop, and $n$ would have been divided by $q$ until it was no longer divisible by $q$. Thus, by the time we check $p$, any smaller prime factors have already been removed from $n$.

2.  \textbf{Handling multiplicity:} The inner \texttt{while} loop, while $n \mod p = 0$, repeatedly divides $n$ by the smallest prime factor $p$ as long as it is divisible. This ensures that all occurrences of this prime factor are found and added to the $factors$ list. Each division $n = n / p$ updates $n$ to a smaller number.

3.  \textbf{The break condition:} The loop terminates when $p > n/p$, which is $p^2 > n$. Consider the value of $n$ when this condition is met. If $n > 1$, suppose $n$ is composite. Then $n$ must have at least one prime factor $q$. Since $n$ is composite, $n = ab$ for some $1 < a \le b < n$. The smallest prime factor $q$ of $n$ must satisfy $q \le a \le \sqrt{n}$. If the loop reaches $p$ such that $p^2 > n$, it means we have checked all integers (and thus all prime numbers) up to $p-1$. If the remaining $n$ were still composite, it would have a prime factor $q \le \sqrt{n}$. Since $p^2 > n$ implies $p > \sqrt{n}$, the factor $q$ must be less than $p$. But if $q$ was a factor of the current $n$, we would have divided by $q$ when the outer loop variable was $q$, contradicting that $q$ is still a factor. Therefore, if $n > 1$ when $p^2 > n$, the remaining value of $n$ must be a prime number itself.

4.  \textbf{Handling the remaining factor:} The final "If $n > 1$, add $n$ to $factors$" block correctly adds this remaining prime factor (if any) to the list.

Combining these points, the algorithm systematically finds and removes the smallest prime factor until the number is reduced to 1 or a prime number, correctly identifying all prime factors with their respective multiplicities. The process terminates because $n$ decreases with each division and $p$ increases in the outer loop, leading to the break condition being met.


This algorithm seems trivial, yet it is actually very challenging to implement in practice. Because it takes huge efforts to find the first $n$ primes, and the algorithm is not efficient. Some more efficient algorithms for the purpose are:
\begin{itemize}
	\item \textbf{Pollard's rho algorithm}: This is a probabilistic algorithm for integer factorization that is particularly effective for large numbers.
	\item \textbf{Elliptic Curve Factorization}: This is another advanced method for integer factorization that uses properties of elliptic curves.
	\item \textbf{General Number Field Sieve (GNFS)}: This is the most efficient classical algorithm for factoring large integers.
\end{itemize}

\section{Prime Testing and Algorithms}
The property of primes brings us to a question: how can we determine whether a number is prime or not? If we want to get many primes, how long will it take? What is the complexity of the algorithm? How can we find the next prime number after a given number?
\defn{Prime Testing}{
	There are several algorithms for testing the primality of a number. Some of the most common ones include:
	\begin{itemize}
		\item \textbf{Trial Division}: This is the simplest method, where we check if a number \( n \) is divisible by any prime number less than or equal to \( \sqrt{n} \). If it is, then \( n \) is composite; otherwise, it is prime.
		\item \textbf{Sieve of Eratosthenes}: This algorithm generates all prime numbers up to a given limit \( n \) by iteratively marking the multiples of each prime starting from 2.
		\item \textbf{Fermat's Little Theorem}: This theorem provides a probabilistic test for primality. It states that if \( p \) is a prime and \( a \) is an integer not divisible by \( p \), then \( a^{p-1} \equiv 1 \mod p \).
	\end{itemize}
}

We will introduce the first two algorithms in detail, and the Fermat's Little Theorem will be introduced later for other purposes.

\subsection{Trial Division}
Trial division is a straightforward and naive algorithm for testing the primality of a number. The idea is simple, we can search for primes linearly from 2 to to any range of numbers we want to check. However, to make sure whether a number is prime or not, we can actually limit the search space to the square root of the number.
\prop{
	This is because if \( n \) is divisible by any number greater than \( \sqrt{n} \), it must also be divisible by a number less than \( \sqrt{n} \).
}

\pf{
	If \( n \) is composite, by the definition of a composite integer, we know that it has a factor \( a \) with \( 1 < a < n \). Hence, by the definition of a factor of a positive integer, we have \( n = ab \), where \( b \) is a positive integer greater than 1. We will show that \( a \leq \sqrt{n} \) or \( b \leq \sqrt{n} \). If \( a > \sqrt{n} \) and \( b > \sqrt{n} \), then \( ab > \sqrt{n} \cdot \sqrt{n} = n \), which is a contradiction. Consequently, \( a \leq \sqrt{n} \) or \( b \leq \sqrt{n} \). Because both \( a \) and \( b \) are divisors of \( n \), we see that \( n \) has a positive divisor not exceeding \( \sqrt{n} \). This divisor is either prime or, by the fundamental theorem of arithmetic, has a prime divisor less than itself. In either case, \( n \) has a prime divisor less than or equal to \( \sqrt{n} \).
}


The algorithm works as follows:

\begin{algorithm}[H]
	\caption{TrialDivision($n$)}
	\begin{algorithmic}[1]
		\Require Integer \(n > 1\)
		\Ensure \textbf{True} if \(n\) is prime, otherwise \textbf{False}
		\For{\(d \gets 2\) to \(\lfloor \sqrt{n} \rfloor\)}
		\If{\(d \mid n\)}
		\State \Return \textbf{False} \Comment{\(n\) is divisible by \(d\), so not prime}
		\EndIf
		\EndFor
		\State \Return \textbf{True} \Comment{\(n\) has no divisors other than 1 and itself}
	\end{algorithmic}
\end{algorithm}
\ex{
	For example, to check if \(29\) is prime, we only need to check divisibility by \(2, 3, 4, 5\) (up to \(\lfloor \sqrt{29} \rfloor = 5\)). Since \(29\) is not divisible by any of these numbers, it is prime.
}

\subsubsection{Complexity of Trial Division}
The time complexity of the trial division algorithm is \(O(\sqrt{n})\) because we only need to check divisibility up to \(\sqrt{n}\). This makes it inefficient for large numbers, especially when \(n\) is very large.

Space complexity is \(O(1)\) since we only need a constant amount of space to store the variables used in the algorithm.

\subsubsection{Correctness of Trial Division}
The proof is straightforward. You may also try to prove it using induction or loop invariant yourself.


\subsection{Sieve of Eratosthenes}
The Sieve of Eratosthenes is an efficient algorithm for finding all prime numbers up to a specified integer \( n \). The algorithm works by iteratively marking the multiples of each prime number starting from 2. The remaining unmarked numbers are primes.
\begin{algorithm}[H]
	\caption{SieveOfEratosthenes($n$)}
	\begin{algorithmic}[1]
		\Require Integer \(n > 1\)
		\Ensure List of prime numbers up to \(n\)
		\State Create a list of integers from 2 to \(n\)
		\For{\(p \gets 2\) to \(\sqrt{n}\)}
		\If{\(p\) is not marked}
		\For{\(k \gets p^2\) to \(n\) with step \(p\)}
		\State Mark \(k\) as composite
		\EndFor
		\EndIf
		\EndFor
		\State Return the unmarked numbers as primes
	\end{algorithmic}
\end{algorithm}

\ex{
	To find all prime numbers up to \(30\):
	\begin{itemize}
		\item Start with a list of numbers from \(2\) to \(30\).
		\item Mark \(2\) and its multiples: \(4, 6, 8, \ldots, 30\).
		\item Mark \(3\) and its multiples: \(9, 12, 15, \ldots, 30\).
		\item Continue this process until reaching \(\sqrt{30} \approx 5.48\).
		\item The remaining unmarked numbers are the primes: \(2, 3, 5, 7, 11, 13, 17, 19, 23, 29\).
	\end{itemize}
}

\subsubsection{Complexity of Sieve of Eratosthenes}
The time complexity of the Sieve of Eratosthenes is \(O(n \log(\log(n)))\), which is significantly faster than trial division for large \(n\). The space complexity is \(O(n)\) because we need to store the list of integers from 2 to \(n\).

We can show the complexity formally:
\lemp{Sieve of Eratosthenes Complexity}{
	The Sieve of Eratosthenes has a time complexity of \(O(n \log(\log(n)))\) and a space complexity of \(O(n)\).
}{
	To analyze the time complexity, consider the algorithm proceeds by iteratively marking the multiples of each prime \(p\) starting from 2 up to \(\sqrt{n}\). For each prime \(p\), the number of multiples marked is approximately \(\frac{n}{p} - 1\).

	Therefore, the total number of marking operations is bounded by
	\[
		T(n) \leq \sum_{\substack{p \leq \sqrt{n} \\ p \text{ prime}}} \left(\frac{n}{p} - 1\right) \leq n \sum_{p \leq \sqrt{n}} \frac{1}{p}.
	\]

	It is a classical result in analytic number theory (see \href{https://en.wikipedia.org/wiki/Mertens'_theorems}{Mertens' theorems}) that the sum of reciprocals of primes up to \(x\) satisfies
	\[
		\sum_{p \leq x} \frac{1}{p} = \log \log x + O(1).
	\]

	By substituting \(x = \sqrt{n}\), we get
	\[
		\sum_{p \leq \sqrt{n}} \frac{1}{p} = \log \log \sqrt{n} + O(1) = \log \frac{1}{2} \log n + O(1) = \log \log n + O(1).
	\]

	Hence, the time complexity satisfies
	\[
		T(n) = O\bigl(n \log \log n\bigr).
	\]

	Regarding space complexity, the algorithm requires \(O(n)\) space to store a boolean array (or similar data structure) representing the primality status of each integer from 2 up to \(n\).

	This completes the proof of the time and space complexity of the Sieve of Eratosthenes.
}



Apart from \href{https://en.wikipedia.org/wiki/Sieve_of_Eratosthenes}{Sieve of Eratosthenes}, \href{https://en.wikipedia.org/wiki/Sieve_of_Atkin}{Sieve of Ktkin} is also a well-known algorithm for finding all prime numbers up to a specified integer \( n \). It is more efficient than the Sieve of Eratosthenes for large values of \( n \).

\section{Greatest Common Divisor and Least Common Multiple}
\subsection{Greatest Common Divisor}
\defn{Greatest Common Divisor}{
	The greatest common divisor (gcd) of two integers \(a\) and \(b\), denoted \(\gcd(a, b)\), is the largest positive integer that divides both \(a\) and \(b\) without leaving a remainder.
}

GCD is a very important concept in number theory. It is used in many algorithms, including the Euclidean algorithm, and solving \href{https://en.wikipedia.org/wiki/Diophantine_equation}{Diophantine equations}, which we will discuss later.

\subsubsection{Coprime and Property of GCD}
You may have realized that prime numbers can be defined in terms of GCD. As some rational numbers $a$ and $b$ are coprime if and only if $\gcd(a, b) = 1$. This means that the only positive integer that divides both $a$ and $b$ is 1. In other words, $a$ and $b$ have no common prime factors.

We can extend the idea of coprimality to more than two numbers.
\defn{Pairwise Coprime}{
	A set of integers \(a_1, a_2, \ldots, a_n\) is said to be pairwise coprime if the gcd of every pair of distinct integers in the set is 1. In other words, for all \(i \neq j\), \(\gcd(a_i, a_j) = 1\).
}

\ex{
	9, 16, 23 are pairwise relatively prime, because gcd$(9,16)=0$, gcd$(9,23)=0$, gcd$(16,23)=0$.
}

If you know basic counting principles, you may realize this is related to the
\href{https://en.wikipedia.org/wiki/Inclusion%E2%80%93exclusion_principle}{Inclusion-Exclusion Principle}.

Determine whether a set of numbers are pairwise coprime or not is not a trivial task. Even though when the numbers are small, we can do this by inspection. But when the numbers are large, we need to use some algorithms to do this. You may tell at a glance that 2, 3, 5, 7 are pairwise coprime, while 4, 6, 8 are not. But how about 10432, 22124, 9342238394?

One common way to check if two numbers are coprime is to iteratively check their prime factorization.

Suppose that the prime factorizations of the positive integers $a$ and
$b$ are $$a=p_1^{a_1}p_2^{a_2}\cdots p_n^{a_n},b=p_1^{b_1}p_2^{b_2}\cdots p_n^{b_n},
$$
where each exponent is a nonnegative integer, and where all primes occurring in the prime factorization of either
$a$ or $b$ are included in both factorizations, with zero exponents if necessary. Then gcd$(a, b)$ is given by
$$\gcd(a,b)=p_1^{\min(a_1,b_1)}p_2^{\min(a_2,b_2)}\cdotp\cdotp\cdotp p_n^{\min(a_n,b_n)},$$
where $\min(x,y)$ represents the minimum of the two numbers $x$ and y.

\ex{
Find $\gcd(100, 250)$.

$$\gcd(120,500)=2^{\min(3,2)}3^{\min(1,0)}5^{\min(1,3)}=2^23^05^1=20.$$
}

\begin{algorithm}[H]
	\caption{GCD Calculation via Prime Factorization}
	\label{alg:gcd_prime_factorization_pseudocode}
	\begin{algorithmic}[1]
		\Procedure{CalculateGCD\_PrimeFactorization}{$a, b$}
		\State \textbf{Input:} Two positive integers $a$ and $b$.
		\State \textbf{Output:} The greatest common divisor $\gcd(a, b)$.
		\State
		\Comment{Step 1: Obtain Prime Factorizations}
		\State Find the prime factorization of $a$: $a \gets p_1^{a_1} p_2^{a_2} \cdots p_k^{a_k}$
		\State Find the prime factorization of $b$: $b \gets q_1^{b_1} q_2^{b_2} \cdots q_m^{b_m}$
		\State
		\Comment{Step 2: Identify Common and Unique Primes}
		\State Let $S$ be the set of all unique prime factors $\{p_1, \dots, p_k\} \cup \{q_1, \dots, q_m\}$.
		\State Let $S = \{r_1, r_2, \dots, r_n\}$ where $r_1 < r_2 < \dots < r_n$.
		\State
		\Comment{Step 3: Standardize Factorizations with all Primes from S}
		\State For each $i \in \{1, \dots, n\}$:
		\State \quad Let $\alpha_i$ be the exponent of $r_i$ in the factorization of $a$ (0 if $r_i$ is not in $a$'s factorization).
		\State \quad Let $\beta_i$ be the exponent of $r_i$ in the factorization of $b$ (0 if $r_i$ is not in $b$'s factorization).
		\State
		\Comment{Step 4: Calculate GCD using Minimum Exponents}
		\State Initialize $\text{gcd\_value} \gets 1$.
		\State For each $i \in \{1, \dots, n\}$:
		\State \quad $\text{min\_exponent} \gets \min(\alpha_i, \beta_i)$
		\State \quad $\text{gcd\_value} \gets \text{gcd\_value} \times r_i^{\text{min\_exponent}}$
		\State
		\Comment{Step 5: Return Result}
		\State \textbf{Return} $\text{gcd\_value}$
		\EndProcedure
	\end{algorithmic}
\end{algorithm}




The time complexity of the algorithm is dependent on the cost of trial division, which is the dominating factor. If we use the naive trial division algorithm, the time complexity is \(O(\sqrt{n})\). However, if we use more advanced algorithms for prime factorization, such as Pollard's rho algorithm or elliptic curve factorization, the time complexity can be significantly reduced.

Below is the proof of the correctness of the algorithm.
\pf{
We need to show that the number $G$ defined by the formula,
$$G = p_1^{\min(\alpha_1, \beta_1)} p_2^{\min(\alpha_2, \beta_2)} \cdots p_n^{\min(\alpha_n, \beta_n)},$$
satisfies the two defining properties of the greatest common divisor:
\begin{enumerate}
	\item $G$ is a common divisor of $a$ and $b$.
	\item Any other common divisor of $a$ and $b$ divides $G$.
\end{enumerate}
By the Fundamental Theorem of Arithmetic, any positive integer greater than 1 can be uniquely represented as a product of prime powers. $a$, $b$, and any of their divisors can be written in this form.

\textbf{Proving that $G$ is a common divisor of $a$ and $b$:}

The prime factorization of $G$ is $G = p_1^{\gamma_1} p_2^{\gamma_2} \cdots p_n^{\gamma_n}$, where $\gamma_i = \min(\alpha_i, \beta_i)$ for each $i$.
To show that $G$ divides $a$, we must verify that for every prime $p_i$, its exponent in the factorization of $G$ is less than or equal to its exponent in the factorization of $a$.
The exponent of $p_i$ in $G$ is $\gamma_i = \min(\alpha_i, \beta_i)$.
The exponent of $p_i$ in $a$ is $\alpha_i$.
By the definition of minimum, we have $\min(\alpha_i, \beta_i) \le \alpha_i$ for all $i = 1, \dots, n$.
Thus, $p_i^{\min(\alpha_i, \beta_i)}$ divides $p_i^{\alpha_i}$ for each $i$.
Since $G$ and $a$ are products of these prime powers, and for each prime the exponent in $G$ is less than or equal to the exponent in $a$, it follows that $G$ divides $a$.
Similarly, to show that $G$ divides $b$, we compare the exponents of $p_i$ in $G$ and $b$.
The exponent of $p_i$ in $G$ is $\gamma_i = \min(\alpha_i, \beta_i)$.
The exponent of $p_i$ in $b$ is $\beta_i$.
By the definition of minimum, we have $\min(\alpha_i, \beta_i) \le \beta_i$ for all $i = 1, \dots, n$.
Thus, $p_i^{\min(\alpha_i, \beta_i)}$ divides $p_i^{\beta_i}$ for each $i$.
Since $G$ and $b$ are products of these prime powers, and for each prime the exponent in $G$ is less than or equal to the exponent in $b$, it follows that $G$ divides $b$.
Since $G$ divides both $a$ and $b$, $G$ is a common divisor of $a$ and $b$.

\textbf{Proving that any common divisor of $a$ and $b$ divides $G$:}

Let $d$ be any positive common divisor of $a$ and $b$. By the Fundamental Theorem of Arithmetic, $d$ can also be written as a product of prime powers. Any prime factor of $d$ must also be a prime factor of both $a$ and $b$. Therefore, the prime factors of $d$ must be a subset of $\{p_1, \dots, p_n\}$.
Let the prime factorization of $d$ be $d = p_1^{\delta_1} p_2^{\delta_2} \cdots p_n^{\delta_n}$, where $\delta_i \ge 0$ for all $i$.
Since $d$ divides $a$, for every prime $p_i$, its exponent in the factorization of $d$ must be less than or equal to its exponent in the factorization of $a$. That is, $\delta_i \le \alpha_i$ for all $i = 1, \dots, n$.
Since $d$ divides $b$, for every prime $p_i$, its exponent in the factorization of $d$ must be less than or equal to its exponent in the factorization of $b$. That is, $\delta_i \le \beta_i$ for all $i = 1, \dots, n$.
Since $\delta_i \le \alpha_i$ and $\delta_i \le \beta_i$, it must be that $\delta_i$ is less than or equal to the minimum of $\alpha_i$ and $\beta_i$. That is, $\delta_i \le \min(\alpha_i, \beta_i) = \gamma_i$ for all $i = 1, \dots, n$.
Now, compare $d = p_1^{\delta_1} \cdots p_n^{\delta_n}$ and $G = p_1^{\gamma_1} \cdots p_n^{\gamma_n}$.
We have shown that for every $i$, $\delta_i \le \gamma_i$.
This means that for every prime $p_i$, its exponent in $d$ is less than or equal to its exponent in $G$.
Therefore, $d$ divides $G$.

\textbf{Conclusion:}

We have shown that $G$ is a common divisor of $a$ and $b$, and that any other common divisor $d$ divides $G$. These two properties uniquely define the greatest common divisor.

Thus, $\gcd(a, b) = p_1^{\min(\alpha_1, \beta_1)} p_2^{\min(\alpha_2, \beta_2)} \cdots p_n^{\min(\alpha_n, \beta_n)}$.
This completes the proof.
}


\subsubsection{Euclidean Algorithm}
In real practice, euclidean algorithm is the most efficient way to compute the GCD of two integers. The Euclidean algorithm is based on the principle that the gcd of two numbers also divides their difference.

Euclidean algorithm is an recursive algorithm for finding the greatest common divisor (gcd) of two integers \(a\) and \(b\). The algorithm is based on two key properties:

\propp{
	\(\gcd(a, b) = \gcd(b, a \mod b)\) for \(b \neq 0\), where $a, b \in \mathbb{Z}$.
}{
	\(\gcd(a, 0) = |a|\).
}{
	To prove this, we can use the definition of gcd. The gcd of two integers \(a\) and \(b\) is the largest positive integer that divides both \(a\) and \(b\). If \(d\) divides both \(a\) and \(b\), then it also divides \(a - kb\) for any integer \(k\). In particular, if we take \(k = \lfloor a/b \rfloor\), then \(d\) divides \(a - kb = a \mod b\). Conversely, if \(d\) divides both \(b\) and \(a \mod b\), then it must also divide \(a\).
	Thus, the gcd of \(a\) and \(b\) is equal to the gcd of \(b\) and \(a \mod b\).
}

\propp{
	\(\gcd(a, b) = \gcd(b, a - kb)\) for any integer \(k\), where $a, b \in \mathbb{Z}$.
}{
	Assume \(d = \gcd(a, b)\). Then \(d\) divides both \(a\) and \(b\), so by definition of divisibility,
	\[
		d \mid a \quad \text{and} \quad d \mid b \implies d \mid (a - kb) \text{ for any integer } k,
	\]
	because $a = k_1d$, $b = k_2d$ for some integers $k_1, k_2$. $a-cb = k_1d - ck_2d = (k_1 - ck_2)d$ for some integer $c$.

	Conversely, if \(d\) divides both \(b\) and \(a - kb\), then it must also divide \(a\). Therefore, the gcd of \(a\) and \(b\) is equal to the gcd of \(b\) and \(a - kb\).
}

\defn{Euclidean Algorithm}{
	Given two non-negative integers $a$ and $b$ (where at least one is non-zero), the algorithm proceeds recursively or iteratively:

	$$ \gcd(a, b) = \begin{cases} |a| & \text{if } b = 0 \\ \gcd(b, a \mod b) & \text{if } b \neq 0 \end{cases} $$
	where $a \mod b$ is the remainder of the division of $a$ by $b$.
}

The algorithm works as follows:
\begin{algorithm}
	\caption{Euclidean Algorithm}
	\label{alg:euclidean}
	\begin{algorithmic}[1]
		\Require Non-negative integers $a$ and $b$ (at least one non-zero).
		\Ensure The greatest common divisor of $a$ and $b$, $\gcd(a, b)$.
		\If{$b = 0$}
		\State \Return $a$
		\EndIf
		\While{$b \neq 0$}
		\State $r \gets a \mod b$ \Comment{Calculate the remainder of $a$ divided by $b$}
		\State $a \gets b$ \Comment{Update $a$ to the original $b$}
		\State $b \gets r$ \Comment{Update $b$ to the remainder $r$}
		\EndWhile
		\State \Return $a$
	\end{algorithmic}
\end{algorithm}

\rmkb{
	In python, there is a pretty handy function \href{https://www.geeksforgeeks.org/divmod-python-application/}{divmod} for handling algorithms related to modulo and division.
}

\ex{
	Find $\gcd(45, 12)$ using the Euclidean algorithm.
	\begin{itemize}
		\item Step 1: $45 \mod 12 = 9$ (since $45 = 3 \times 12 + 9$)
		\item Step 2: $12 \mod 9 = 3$ (since $12 = 1 \times 9 + 3$)
		\item Step 3: $9 \mod 3 = 0$ (since $9 = 3 \times 3 + 0$)
		\item Step 4: Since the remainder is now zero, we stop here. The gcd is the last non-zero remainder, which is \(3\).
	\end{itemize}
	Thus, $\gcd(45, 12) = 3$.
}

\propp{
	The time complexity of the Euclidean algorithm is \(O(\log(\min(a, b)))\).
}{
	We can prove the complexity by counting the cost of each iteration and the number of iterations. Inside each iteration, we perform only basic arithmetic operations (addition, subtraction, and modulo), which take constant time O(1).

	The number of iterations in finding $\gcd(a, b)$ can be expressed by
	$$
		\lceil \log_2(\min(a, b)) \rceil.
	$$
	This is because we always need to put the smaller number in the second position, and in each iteration, the smaller number is reduced by at least half. $b = 1$ if and only if the algorithm reaches the base case, else, we always have $b \geq 2$. Hence, in the worst case, $a$ is continually halved, that will be $a = 2^k$ for some integer $k$. The number of iterations is at most $k$, which is $\lceil \log_2(\min(a, b)) \rceil$.

	By the definition of big O notation,
	$$
		\exists c > 0 \text{ and } n_0 \in \mathbb{N} \text{ such that } T(n) \leq c\log_2(n) \text{ for all } n \geq n_0.
	$$
	Thus, we can conclude that the time complexity of the Euclidean algorithm is \(O(\log(\min(a, b)))\).
}


\prob{
	Let $n$ be a positive integer greater than 1. Consider the set
	$$\mathbb{Z}_n^* = \{ a \in \{0, 1, \dots, n-1\} \mid \gcd(a, n) = 1. \}$$
	\bigskip
	We define a binary operation on $\mathbb{Z}_n^*$ as multiplication modulo $n$. Prove that $(\mathbb{Z}_n^*, \cdot_n)$ is a group.
	(You will need to show closure under multiplication modulo $n$, associativity, existence of an identity element, and existence of an inverse for each element. The existence of an inverse relies on properties of the greatest common divisor, specifically Bezout's identity).
}

\subsection{Least Common Multiple}
\defn{Least Common Multiple}{
	The least common multiple of the positive integers $a$ and $b$ is the smallest positive integer that is
	divisible by both $a$ and $b$. The least common multiple of $a$ and $b$ is denoted by lcm$(a, b)$.
}

Suppose the prime factorizations of \(a\) and \(b\) are given by
\[ a = p_1^{a_1} p_2^{a_2} \ldots p_n^{a_n}, \quad b = p_1^{b_1} p_2^{b_2} \ldots p_n^{b_n}, \]
where \(a_i\) and \(b_i\) are the exponents of the prime factors of \(a\) and \(b\), respectively.

The product \(ab\) can be expressed as a prime factorization:
\[ ab = p_1^{a_1+b_1} p_2^{a_2+b_2} \ldots p_n^{a_n+b_n}. \]

From this, we can deduce that any common multiple of \(a\) and \(b\) must be a product of their prime factors raised to at least the maximum exponent found in either \(a\) or \(b\). Therefore, the \(\text{lcm}(a, b)\) can be expressed as
\[ \text{lcm}(a, b) = p_1^{\max(a_1,b_1)} p_2^{\max(a_2,b_2)} \ldots p_n^{\max(a_n,b_n)}. \]
This ensures that \(\text{lcm}(a, b)\) is divisible by both \(a\) and \(b\), and it is the smallest such number with this property.

\ex{
Find lcm$(120, 500)$.
$$\text{lcm}(120,500)=2^{\max(2,3)}3^{\max(1,0)}5^{\max(1,3)} = 8\times 3 \times 125= 3000$$
}


Have you noticed, that $ab$ is actually the product of $\text{lcm}(a,b)$ and $\gcd(a,b)$? Becuase the order of each
term in the prime factorization of $ab$ is a sum of two number, what ever which number is the maximum of the minimum,
we always have the order of $a_n+b_n$. For 120 and 500 we have
$$ab = 120\times 500 = 60000 = \text{lcm}(a,b)\times \gcd(a,b) = 3000\times 20$$

\thm{
	Integer Product as GCD and LCM}{Let $a$ and $b$ be positive integers. Then $ab= \gcd ( a, b) \cdot $lcm$( a, b).$
}


\subsection{GCD as Linear Combination}
\defn{Linear Combination}{
	A linear combination of integers \(a\) and \(b\) is an expression of the form \(ax + by\), where \(x\) and \(y\) are integers.

	More generally for \(n\) integers \(a_1, a_2, \ldots, a_n\), a linear combination is an expression of the form
	\[
		c_1 a_1 + c_2 a_2 + \ldots + c_n a_n,
	\]
	where \(c_1, c_2, \ldots, c_n\) are integers.
}

The significance of GCD in number theory is accompanied by its representation as a linear combination of integers. This is known as Bézout's identity, or Bézout's lemma. This result is crucial in solving many problems in number theory, including Diophantine equations.

\thm{Bézout's Identity}{
	For any integers \(a\) and \(b\), there exist integers \(x\) and \(y\) such that
	\[
		\gcd(a, b) = ax + by.
	\]
}

Bézout's identity allows us to implement extended Euclidean algorithm, which not only computes the GCD of two integers but also finds the coefficients \(x\) and \(y\) such that \(ax + by = \gcd(a, b)\).
\ex{
	Let $a = 1071$, $b = 462$.
	$$
		\begin{aligned}
			1071 & = 2 \cdot 462 + 147 \\
			462  & = 3 \cdot 147 + 21  \\
			147  & = 7 \cdot 21 + 0    \\
		\end{aligned}
	$$
	Back-substitute:
	$$
		\begin{aligned}
			21          & = 462 - 3 \cdot 147 \quad \text{(from step 2)}  \\
			147         & = 1071 - 2 \cdot 462 \quad \text{(from step 1)} \\
			\implies 21 & = 462 - 3 \cdot (1071 - 2 \cdot 462)            \\
			            & = 7 \cdot 462 - 3 \cdot 1071.
		\end{aligned}
	$$
	Thus, $\gcd(1071, 462) = -3 \cdot 1071 + 7 \cdot 462$.
}

\chapter{Modular Equations and Diophantine Equations}
\section{Solving Linear Congruence}
\subsection{Modular Inverse}
\section{Diophantine Equations}
\subsection{Linear Diophantine Equations}
\section{System of Linear Congruence}
\subsection{Chinese Remainder Theorem}
\section{Fermat's Little Theorem}

\chapter{Applications of Number Theory}
\section{Cryptography}
\subsection{Basics of Cryptosystems}
\subsection{Diffie-Hellman Key Exchange}
\subsubsection{RSA Algorithm}
\section{Hash Functions}
\subsection{Modular Hash Functions}
\subsection{Multiplicative Hash Functions}
\subsection{Polynomial Hash Functions}
\subsection{SHA-256}
\section{Representation of Integers}
\subsection{Modular Exponentiation Algorithm}
\subsection{Base Representation and Conversion}

\nocite{*}
\printbibliography

\end{document}
